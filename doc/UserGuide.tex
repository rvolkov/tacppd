%% LyX 1.4.1 created this file.  For more info, see http://www.lyx.org/.
%% Do not edit unless you really know what you are doing.
\documentclass[english]{article}
\usepackage[]{fontenc}
\usepackage[latin1]{inputenc}
\usepackage{graphicx}

\makeatletter

%%%%%%%%%%%%%%%%%%%%%%%%%%%%%% LyX specific LaTeX commands.
%% Because html converters don't know tabularnewline
\providecommand{\tabularnewline}{\\}
%% A simple dot to overcome graphicx limitations
\newcommand{\lyxdot}{.}


%%%%%%%%%%%%%%%%%%%%%%%%%%%%%% Textclass specific LaTeX commands.
\newenvironment{lyxcode}
{\begin{list}{}{
\setlength{\rightmargin}{\leftmargin}
\setlength{\listparindent}{0pt}% needed for AMS classes
\raggedright
\setlength{\itemsep}{0pt}
\setlength{\parsep}{0pt}
\normalfont\ttfamily}%
 \item[]}
{\end{list}}

\usepackage{babel}
\makeatother
\begin{document}

\title{TACPPD User Guide}


\author{\includegraphics[bb = 0 0 200 100, draft, type=eps]{tacppd.org.jpg}}


\author{(c) Copyright 2003-2006 by Roman Volkov and contributors}

\maketitle
Special thanks to Brett Maxfield.


\section{Contacts}

See tacppd home site http://tacppd.org. For comments, issues and feature
requests you can use tacppd sourceforge resources linked to http://tacppd.org
site. For other information you can contact directly with Roman Volkov,
rv@tacppd.org. For some extra information and services, commercial
support and commercial developing contact with http://tacppd.com.


\section{Send us your comments}

Tacppd developers welcome your comments and suggestions on the quality
and usability of this guide.

\begin{enumerate}
\item Have you found any errors?
\item Is the information clear?
\item Do you need more informaton?
\end{enumerate}

\section{Preface}


\subsection{Organization}

This guide is organized as follows:


\subsection{Related Documentation}

Available documentation:

\begin{enumerate}
\item TACACS+ protocol.
\item TACPPD programmer guide
\end{enumerate}

\subsection{Conventions}

\begin{enumerate}
\item conventions in text
\item conventions in examples
\item conventions in operating system
\item conventions in Cisco configuration
\end{enumerate}

\section{COPYRIGHT info}

The tacppd (tacacs++ server daemon) software is Copyright (C) 1998-2005
by tacppd team and contributors. For more information visit tacppd
home site http://tacppd.org. You may use, distribute and copy the
tacppd software under the terms of GNU General Public License version
2. See COPYING file in tacppd distribution for text of GNU GPL. You
should have received a copy of the GNU General Public License along
with this program; if not, write to the Free Software Foundation,
Inc., 59 Temple Place, Suite 330, Boston, MA 02111-1307 USA. This
product includes software developed by Eric Young. This product includes
software developed by Alec Peterson. It uses RSA Data Security, Inc.
MD5 Message-Digest Algorithm. Also some copyright information for
other parts available inside source code (NET-SNMP library code).


\section{SUPPORT info}

This product has community support, available via http://tacppd.org
resources: mailing list, phorum, bugtracking and feature requests
system from SourceForge's tacppd page (http://sourceforge.net/progects/tacppd).
You should know, that support will be provided only when community
people will have free time and possibilities to do it, so please,
don't require a lot. For information about commercial support visit
http://tacppd.com, but you should know, that support will not be provided
under any circumstances for this program by tacppd.com, it's employees,
volunteers or directors, unless a commercial support agreement and/or
explicit prior written consent has been given.


\section{Introduction}

This is AAA server (authentication,authorization,accounting) for network
devices with extra features. The main goal - full database support
+ integration with bi lling system. Also we try to do mechanism for
real-time user sessions control with the really true information for
it (if you have network with some untrusted channels , you can loss
accounting packets from network devices). We are using model \char`\"{}all-i
n-one\char`\"{} (tacacs+ with telnet server, with http server, with
NetFlow collector, with SNMP poller and other) for make it more easy
for installation and using, also for possible integration in hardware
box with telnet/web control and AAA server functionality . This program
is developing on C++. It can be compiled for any {*}nix system with
POSIX threads (I suppose). Currently supported platforms only Linux(x86),
FreeBS D(x86), Solaris8/9(SPARC) (I don't have access to other, but
if you need in other, conta ct me, may be i'll help). Most information
about network devices valid only for Cisco e quipment. For your device,
see vendor documentation. Now i am thinking about raise tacppd to
some set of API or may be separate libra ries for doing applications
with integrated CLI, http, log and debug. Network applications which
can be easy integrated to hardware linux-based boxes and use external
storage like iSCSI devices. For example, smtp/pop3 server, http server,
etc. Now i am working for application-independend core of tacppd,
which can be separated from tacacs+ or a ny other application code
to separated API or library. Hope, you will help me and some later
i'll create several tacppd's code based projects of several network
services servers. In more recent plans - raise tacppd to network management/accounting
system.


\subsection{Abstract}

TACACS+ (Terminal Access Control Access Server Plus) provides access
control for routers, network access servers and other networked computing
devices via one or more cen tralized servers. TACACS+ provides separate
authentication, authorization and accounting services. The full describe
of TACACS+ protocol you can find at ftp://ftp-eng.cisco.com/pub/tacacs/tac-rfc.1.78.txt
or http://search.ietf.org/internet-drafts/draft-grant-tacacs-02.txt
or in /doc directory of this distribution. TACACS+ is used mostly
for Cisco devi ces (but I saw it support in 3Com HiPer too, and read
about it in Cabletron devices) . RADIUS - this is other similar protocol,
which working throught UDP and have way to use vendor specific attributes.
RADIUS - more compatible protocol but less functional, may be later
we will do R ADIUS support in tacppd.


\subsection{Features}

\begin{enumerate}
\item SQL database support for store user information (together with authorization
information, like per-client IP addresses, etc)
\item multiply redudancy databases, can connect to several types of databases
if appropriate module available
\item open interface for create own database support modules
\item password field in database encrypted by MD5
\item internal IP address pooling system
\item personal tacacs+ key for each device + access control 
\item reconfiguration and monitoring on-fly via telnet 
\item per-users work time-plan
\item NAS control via SNMP
\item SNMP modules with open interface for easy new devices adding
\item users in database control via telnet command-line interface
\item intergated NetFlow data collector with some aggregating functionality
and logging directly to database
\item accounting logging directly to database
\end{enumerate}

\subsection{ToDo}

\begin{enumerate}
\item Universal modular billing system 
\item peer-to-peer roaming 
\item Monitoring and configuration via http 
\item RADIUS support 
\item SNMP agent support 
\end{enumerate}
And much more. Use feature requests in tacppd's SourceForge project
page and also write to phorum/mailing lists for expand it.


\section{Directories and files}

\begin{lyxcode}
/contrib~-~contributed~products

/etc~-~config~dir~

~tacppd.conf.ex~-~example~configuration~file~

/db~-~database~modules~

~pgsql.so~-~PostgreSQL~driver~

~mysql.so~-~MySQL~driver~

~msql.so~-~miniSQL~driver~

~none.so~-~void~device~

/snmp~-~snmp~devices~modules~

~cisco.so~-~Cisco~devices~support~

~none.so~-~void~device~

/include~-~include~files~for~database~and~snmp~modules~

/run~-~pid-file~dir~

/log~-~log-file~dir~

/tmp~-~temporary~files~directory~

/ext~-~external~programs~dir~

/http~-~http~monitor~files~

/resrc~-~resource~modules~dir~

tacppd\_start.sh.example~-~example~daemon~start~script~

tacppd~-~tacppd~bin~file~
\end{lyxcode}

\section{Source compilation}

If you will set from binary distribution, skip this chapter. Fast
guide available in file INSTALL inside tacppd distribution. 


\section{Installation}

1. Get binary build from ftp or http for appropriate platform, consult
http://tacppd.org for downloads.

latest build from nightly builds:

tacppd-linux-x86-BUILDDATE.tgz

tacppd-linux-x86-snmpstatic-BUILDDATE.tgz

tacppd-fbsd-x86-BUILDDATE.tgz

tacppd-fbsd-x86-snmpstatic-BUILDDATE.tgz

tacppd-solaris-SPARC-BUILDDATE.tgz

tacppd-solaris-SPARC-snmpstatic-BUILDDATE.tgz

where BUILDDATE - this is build date in YYYYMMDD form

latest release build: 

tacppd-linux-x86-VERSION.tgz

tacppd-linux-x86-snmpstatic-VERSION.tgz

tacppd-fbsd-x86-VERSION.tgz

tacppd-fbsd-x86-snmpstatic-VERSION.tgz

tacppd-solaris-SPARC-VERSION.tgz

tacppd-solaris-SPARC-snmpstatic-VERSION.tgz

where VERSION - build version in form 0.0.0

The word \char`\"{}snmpstatic\char`\"{} in the filenames above means
that those binaries have been statically linked with the Net-Snmp
package. To use the dynamic Net-Snmp library, you should get Net-Snmp
from it's homepage at http://www.net-snmp.org and install it. Read
through the net-snmp documentation, configure, compile and install
the net-snmp shared library using the --enable-shared configure option. 

The default for net-snmp is to only create a static library. When
you have compiled the snmp shared library, you can then download and
use the tacppd package without the \char`\"{}snmpstatic\char`\"{}
word. 


\subsection{New install}

For first-time install copy file \char`\"{}tacppd\_start.sh.orig\char`\"{}
to \char`\"{}tacppd\_start.sh\char`\"{} and edit it for your directory
(read comments in file). For first-time install copy file \char`\"{}etc/tacpp.conf.orig\char`\"{}
to \char`\"{}etc/tacpp.conf\char`\"{} and review it (and read \char`\"{}etc/README\char`\"{}).
Check if you have workable database libraries (database modules require
it). For it, check database module which you will use by ldd: 

ldd db/postgresql.so 

ldd db/mysql.so

if you see library libpq.so/libmysqlclient.so inside - all right in
other case, check you /etc/ld.so.conf or LD\_LIBRARY\_PATH for path
to appropriate shared library (and your PostgreSQL/MySQL too) 

REMEMBER! you MUST HAVE appropriate PostgreSQL or MySQL libraries
(or miniSQL, if you will use this driver), if you plan to use Oracle,
than you m ust have Oracle client installed and configured. 


\subsubsection{Database System}

Please, contact your database administrator for database name and
database user login/password information.

Edit \char`\"{}etc/tacpp.conf\char`\"{} file for database, REMEMBER
database key, if you have lost it, no way to restore user passwords!
The crypto algorythm uses 2 keys - static - this is this key, and
variable key - the user login. If you change username, than you will
lost your password! Never use UPDATE request for change username!
Please, use tacppd CLI for changes in database tables. For change
username, delete this user and add new. If you will use key \char`\"{}none\char`\"{},
than no encryption will be done. Use it if you plan to use external
database management system, or if this system does not supports tacppd-style
encryption. 

Before use, you should create database tables. You MUST use tacppd
internal database manipulation manager for it (the database tables
format can be changed during development and it should create mostly
modern table views). If you already have database tables for tacppd,
review ReleaseNotes, CHANGES and this README file for changes. After
tacppd starts, you should use CLI (command line interface): in \char`\"{}database\char`\"{}
mode you should use command \char`\"{}create\char`\"{} for tables
creation (see later for CLI description).

Indexes will be created automatic. See error logs in case of errors.

Now you can add users into database. Tacppd has simple internal user
manipulatio n resource (see later for CLI documentation). If you will
use own database manipul ation software, remember, that you should
encrypt passwords before insert. For example, you can use contributed
Perl script in /contrib dir for crypt/decrypt passwords. 

for PostgreSQL (http://www.postgresql.org): contact database administrator
to create special user with grants to create, update, insert and select
tables and indexes. The next st ep - from database admin user run
psql like this: psql -U username database\_name and type password
on request try to select/create tables for check your rights

for Oracle (http://www.oracle.com): change to oracle user (typically
by command \char`\"{}su - oracle\char`\"{}) and run: sqlplus username/password@your.oracle.service
and test it. We recommend to contact your DB administrator before.

BE CAREFULL!!! The Database table format now in developing and can
be changed from build to build with notifications only in CHANGES
file!!!

I plan to freeze database formats after first BETA version, but now
i highly recommend to use CLI interface for database data management!

From 0.0.5 development i add database version number to name of table,
but the version will be frozen only after release. 


\subsubsection{Before start steps}

1. check tacppd/tacppd\_start.sh file for right path to tacppd home
(read throught it for several options) 

2. edit tacppd/etc/tacpp.conf file (copy it from tacpp.conf.orig)
read carefully throught it and throught etc/README file the default
admin user have username: admin and password: admin\_password please,
change database ip and username/password NOTE: remember database crypto
key!!! You will have lost your client passwords if you have lost database
key! Otherwise, use \char`\"{}none\char`\"{} key for unencrypted password
fields. 

3. check availability of database be sure, that your user created
in database and firewalls configuration correct 

4. start tacppd\_start.sh script 

5. try to connect to control system by telnet localhost 2222 (or any
port you defined in configfile for telnet) you should see username
request type admin and type admin\_password as password (this is default
setting in tacpp.conf.orig) Feel free to change default settings.

in CLI (command line interface) you can use ?\textless{}CR\textgreater{}
for getting help about available commands. {[}command] ?\textless{}CR\textgreater{}
- help about command arguments


\subsection{Upgrade}

Upgrade procedure require stop current tacppd by command {}``killall
tacppd'', unpack distribution to place of old tacppd (configuration
file and shell scripts will not be changed, will be replaced files
etc/tacppd.conf.ex and tacppd\_start.sh.example), and run tacppd again.
Read release notes for critical changes.


\section{Configuration}

Tacppd configuration stored in file tacppd.conf. Distribution has
file etc/tacppd.conf.ex, just copy this file to etc/tacppd.conf. This
file you can edit before tacppd start. During tacppd work you have
to use command-line interface (CLI) via telnet for tacppd configuration.

the configuration file latest notes can be available in etc/README
(see it before you will start create configuration)

tacppd config file - etc/tacpp.conf, with distribution we are provide
file etc/tacpp.conf.orig with common examples, you can write own tacpp.conf
file before start or customize tacpp.conf.orig (copy tacpp.conf.orig
to tacpp.conf before in this case). You can edit this file from editor,
but be carefully: edit this file BEFORE starting tacppd - if you will
edit it AFTER, than you can simple lost your changes if you will do
\char`\"{}write\char`\"{} command from CLI. Remember - you need in
manual config file edit only in first-time installation. Or if you
will do anything without working tacppd. In other case you should
use CLI (command line interface) commands for configuration - the
changes will apply after you press \textless{}Enter\textgreater{}.

Tacppd like Cisco device has 2 configurations - one running-config,
and other - stored config (or startup-config). The startup-config
for tacppd - file etc/tacp p.conf, running-config - in memory. You
can't reload startup-config during tacppd work, but you can change
running-config and copy it to startup-config.

the config file use next things: remark - string, which begins with
pund (\#) symbol first command should be word \char`\"{}config\char`\"{},
last - word \char`\"{}exit\char`\"{}

The configuration file supports all commands which you can use in
CLI \char`\"{}config\char`\"{} mode. 

there are several sections available: debug access manager listener
pool nas database bundle billing peer 


\subsection{debug}

Output debug information. Debug information output to file log/debug.log.

\begin{lyxcode}
!~debug

debug~file

~facility~db

~facility~pool

~facility~billing

~facility~track

~facility~tacacs

~facility~aaa

~facility~snmp

~facility~server

~exit
\end{lyxcode}
Facilities mean output debug for such operation.

output of debug information file - output to file syslog - output
to local syslog \textless{}ip addr\textgreater{} - output to external
syslog facility - this is type of information which will be outputed
to this destination 


\subsection{access lists}

Access control. Use Regular Expressions syntax. All connections in
tacppd controlled via access lists. It used for control TCP and UDP
connections, also it control per-user manager access.

\begin{lyxcode}
!~access~list

access~1

~permit~\textasciicircum{}127.0.0

~permit~\textasciicircum{}192.168.0

~exit

access~2

~permit~\textasciicircum{}127.0.0

~deny~\textasciicircum{}192.168.2.3\$

~permit~\textasciicircum{}192.168.2

~exit

access~3

~permit~\textasciicircum{}127.0.0

~permit~\textasciicircum{}192.168.0

~permit~\textasciicircum{}172.17.35

~exit
\end{lyxcode}
Access list checks from begin to end. If any rule permit something,
than no more checks. If any rule deny, no more checking too. If no
any match, than it mean deny.

access list - control access to listeners and for users \char`\"{}access
1 permit \textasciicircum{}127.0.0\char`\"{} 1 - number of access
list (1-32000) \textasciicircum{}127.0.0 - regular expression you
can use usual regular expressions, like: \textasciicircum{}127.0.0.1\$
and so on 


\subsection{manager user}

Administrators. Users for access via telnet or web-interface to tacppd.
Also users for access across TCI (tacppd control interface).

tacppd manager user user user\_name password 'cleartext password'
cpassword 'md5 digest password' type terminal ! terminal or http available
acl access\_list\_number exit

!NOTE! - during 'write' command password will be stored as md5 digest 


\subsection{listener}

Tcp/Udp listeners with several sevices.

configuration of several listeners listener port\_number ! if you
run it not from root, ! then tcp/udp port number can't be less 1024
type \textless{}type\textgreater{} ! which server maxconnect 100 !
maximum connection number acl access\_list\_number exit

the listener types available: terminal - telnet for CLI tacacs - tacacs+
radius - radius radacct - radius accounting http - http interface
netflow - NetFlow collector bundle - for tacppd bundle peer - for
roaming peers snmp - snmp agent

!NOTE! - maximum connection number must be reasonable, this is limits
in case of burst load. !NOTE! - radius,radacct,bundle,peer,snmp -
not implemented while 


\subsection{pool}

address pool system pool address\_pool\_number address.from.this.pool
the addresses you can write in form like: 192.168.0.1-192.168.0.10
(this is addresses from 1 to 10) or as 192.168.0.2,192.168.0.5 (separate
2 addresses) more complex example: 192.168.0.1-192.168.0.5,192.168.0.7,192.168.0.9-192.168.0.12 


\subsection{network device}

Network devices which tacppd must control. Usually this is Cisco routers/access
servers.

network devices support

nas \textless{}ip.addr.of.device\textgreater{} description \textless{}your\_comment\_about\_this\_device\textgreater{}
tacacskey \textless{}tacacs\_plus\_key\textgreater{} radiuskey \textless{}radius\_key(for
future implementation)\textgreater{} snmpcommunity \textless{}your\_snmp\_community\textgreater{}
loginstring \textless{}string\_for\_your\_nas\textgreater{} pwdstring
\textless{}password\_invite\_string\textgreater{} defaultauthorization
yes\textbar{}no module cisco.so polldelay 5 trfcounter no shutdown
no 


\subsection{database}

Databases list. Can be several databases. First in list - main database.
During work tacppd try to connect to first database and if connect
will refused it will conecct to second. And so more. Updates, Inserts
and other changes to database will be doing to all databases. There
no more things for maintain databases equality. May be it will do
by some external tools.

\begin{lyxcode}
!~database~~

database~tacppddb@192.168.0.9:5432

~cryptokey~'none'

~login~tacppd

~password~TacppdPwd

~module~pgsql.so

~shutdown~no

~exit
\end{lyxcode}
String {}``tacppd@192168.0.9:5432'' means that database server on
host 192.168.0.9 on port 5432 use database named {}``tacppddb''.
Cryptokey used for crypt passwords in database. Key = {}``none''
- unencrypted (for debugging). In any case crypted and cleartext passwords
in database coded to Base64.

work with database system BE CAREFUL with database cryptokey!!! If
you lost key, you lost your database!!! Or use \char`\"{}none\char`\"{}
key to disable encryption.

database db\_name@db\_host:db\_port cryptokey \textless{}key\textgreater{}
login \textless{}db\_login\textgreater{} password \textless{}db\_pwd\textgreater{}
module \textless{}database driver\textgreater{} shutdown no 


\subsection{bundle}

Information for create tacppd bundle. There are only one bundle possible.

\begin{lyxcode}
bundle~239.1.1.1:11000

~cryptokey~key

~priority~1

~ttl~1

~exit
\end{lyxcode}
If you set cryptokey to {}``none'' than bundle traffic will be unencrypted
(for debug purposes). Ttl value - this is multicast time-to-live in
hops. Ttl=1 for multicasts in one LAN, more value - if you have routers
between servers in one bundle.

communicate tacppds to bundle

bundle \textless{}bundle\_ip\textgreater{}:\textless{}bundle\_port\textgreater{}
description \textless{}desc\textgreater{} cryptokey \textless{}key\textgreater{}
ttl \textless{}multicast ttl value (usually 1)\textgreater{} 


\subsection{billing}

billing modules descriptions

billing \textless{}number\textgreater{} description \textless{}descr\textgreater{}
module \textless{}module.so\textgreater{} shutdown no 


\subsection{peer}

Peers to other tacppds for roaming.

roaming peers

peer \textless{}ip\textgreater{}:\textless{}port\textgreater{} description
\textless{}descr\textgreater{} cryptokey \textless{}key\textgreater{}
timeshift \textless{}+/- time\textgreater{} mask \textless{}regular
expr\textgreater{} mask \textless{}regular expr\textgreater{}


\section{Tacppd CLI - command-line interface}

command-line interface available throught telnet. You can configure
your server throught CLI, and changes will be immediately available.
The next issue - \char`\"{}write\char`\"{} command will crypt cleartext
passwords and write current configuration to configuration file \char`\"{}etc/tacpp.conf\char`\"{}.
When you connect to server via telnet (\char`\"{}telnet ip.addr port\char`\"{}
- ip.add - addr of host where tacppd work, port - portnumber in tacpp.conf)
after user access verification (default user=admin password=admin\_password)
you see string: your\_hostname\#\char`\"{} it means, that you now
in CLI. try write \char`\"{}?\char`\"{} and press \textless{}ENTER\textgreater{}
for help No edit keys available in this version, be carefull! I have
tested telnet special codes only for Linux telnet, from Windows's
HyperTerm you can have problems. Currently you can see and change
configuration and check/drop users on your access servers 

\begin{lyxcode}
command~map:~tacpp(hostname)\#~?~config~-{}-{}-{}-{}-~~~~~~~~~~~~~~~~~~~~~~~~~~~~~~~~~debug~~~~~~~~~~~~~~~~~~~~~~~~~~~~~~~~~manager~~~~~~~~~~~~~~~~~~~~~~~~~~~~~~~~~listener~~~~~~~~~~~~~~~~~~~~~~~~~~~~~~~~~access~~~~~~~~~~~~~~~~~~~~~~~~~~~~~~~~~pool~~~~~~~~~~~~~~~~~~~~~~~~~~~~~~~~~nas~~~~~~~~~~~~~~~~~~~~~~~~~~~~~~~~~database~~~~~~~~~~~~~~~~~~~~~~~~~~~~~~~~~bundle~~~~~~~~~~~~~~~~~~~~~~~~~~~~~~~~~billing~~~~~~~~~~~~~~~~~~~~~~~~~~~~~~~~~peer~~~~~~~~~~~~~~~~~~~~~~~~~~~~~~~~~show~~~~~~~~~~~~~~~~~~~~~~~~~~~~~~~~~exit~~~~~~~~~~~~!~exit~from~configuration~sectio~write~~~~~~~~~~~~~~~~~~~~~~~~~~~~~~~~~~~~~~~~~~~!~write~current~configuration~show~-{}-{}-{}-{}-{}-{}-~~~~~~~~~~~~~~~~~~~~!~show~~~~~~~~~~~~~~~~~~~~~~~~~~~~~~~~~users~~~~~~~~~~~!~show~usertable~~~~~~~~~~~~~~~~~~~~~~~~~~~~~~~~~nas~~~~~~~~~~~~~!~show~nas~information~clear~-{}-{}-{}-{}-{}-~~~~~~~~~~~~~~~~~~~~!~clear~information~~~~~~~~~~~~~~~~~~~~~~~~~~~~~~~~~user~~~~~~~~~~~~!~drop~user~

~~~~~~~~~~~~~~~~~~~~~~~~~~~~~~~port~~~~~~~~~~~~!~drop~nas~port~database~-{}-{}-{}-~~~~~~~~!~database~configuration~section~~~~~~~~~~~~~~~~~~~~~~~~~~~~~~~~~create~~!~create~database~tables~~~~~~~~~~~~~~~~~~~~~~~~~~~~~~~~~add~~~~~~~~~~~~~!~add~entry~to~database~~~~~~~~~~~~~~~~~~~~~~~~~~~~~~~~~del~~~~~~~~~~~~~!~delete~entry~from~database~~~~~~~~~~~~~~~~~~~~~~~~~~~~~~~~~show~~~~~~~~~~~~!~show~data~~~~~~~~~~~~~~~~~~~~~~~~~~~~~~~~~modify~~!~modify~entry~in~database~~~~~~~~~~~~~~~~~~~~~~~~~~~~~~~~~exit~~~~~~~~~~~~!~exit~form~database~section~debug~~-{}-{}-{}-{}-~~!~debug~mode~exit
\end{lyxcode}
config - change server configuration you should do \char`\"{}write\char`\"{}
command to store your working configuration to startup configuration
file (etc/tacpp.conf) (use commands from configuration file documentation
upper)

show - show several information currently you can see only currently
working users table and network devices under control

clear - clear anything currently you can drop users/ports on network
devices 

database - database control mode


\section{Tacppd HTTP interface}

Tacppd has HTTP interface. In default configuration the http server
started on port 8888, and use username/password \char`\"{}webmin/webmin\char`\"{}
for access. Try it by http://localhost:8888 check access rights in
config before. Tacppd has inbuilt web-server with Perl support.


\section{Cisco device configuration}

Here only some simple information how to configure Cisco devices to
use tacppd. You have to be experiencing Cisco user. If you do not
clearly understand what we will do in configuration, read Cisco documentation
at http://www.cisco.com.

you should setting up TACACS+ on NAS: aaa new-model aaa group server
tacacs+ OUR\_TACACS server your.server.ip.addr tacacs-server host
your.server.ip.addr port server\_port key tacacs\_key

NOTE: port and key should be same as in tacppd configuration file
NOTE2: \char`\"{}aaa group server\char`\"{} command can not be available.
Don't care.

-for authentication add: aaa authentication login DIALUP group OUR\_TACACS

-for ppp authentication: aaa authentication ppp DIALUP if-needed group
OUR\_TACACS

-for ppp authorization: aaa authorization network DIALUP group OUR\_TACACS 

-for tty commands authorization: aaa authorization exec DIALUP group
OUR\_TACACS

-for per-command authorization: aaa authorization commands \textless{}privilege
level 1-15\textgreater{}

-for accounting: aaa accounting suppress null-username aaa accounting
update newinfo aaa accounting network DIALUP start-stop group OUR\_TACACS

-you can also use accounting periodic updates for extra users list
checking: aaa accounting update periodic 1

-and configure this all on interfaces: ppp authentication chap pap
DIALUP ppp authorization DIALUP ppp accounting DIALUP 

-and on tty lines: authorization exec DIALUP accounting connection
DIALUP login authentication DIALUP

Next thing - set up SNMP: access-list 2 permit host your\_server\_host
snmp-server community your\_community\_string RW 2 BE CAREFULL - add
only your tacppd host into SNMP access-list!


\subsection{tacacs+ configuration}

First, you should configure tacacs+ servers information. 


\subsection{AAA configuration}

AAA system.


\subsection{SNMP configuration}

For use SNMP polling tacppd feature.


\section{Database configuration}

In any case you have to read vendor documentation and consult with
your database administrator. Our information only about some specific
things.


\subsection{PostgreSQL}

Available from http://www.postgresql.org. Has embedded procedural
language PL/PGSQL and also functionality of {}``big'' system. Currently
i use PostgreSQL for tacppd developing/testing and this is most supported
database.


\subsection{MySQL}

Very useful and fast, but with restricted functionality.


\subsection{MiniSQL}

Very small and restricted.


\subsection{Oracle}

You have to have Oracle, or if you will using it for developing process
you can get it with developer license ftom OTN: http://otn.oracle.com.


\section{Information in database}

NOTE: database tables name and it's fields can be changed in any time.
I plan to freeze changes in it only in first beta-version.

database table descriptions available in TacDb.cc file

For understanding information stored into database, you should be
familiar with information, used in tacacs+ and by network devices.


\subsection{authentication}

check user access rights

TABLES: tacpp\_v1\_usr - user information 

tacpp\_v1\_acc - user access rights 

tacpp\_v1\_add - additional user data


\subsubsection{user information}

store username, password, open/close date and time, several groups
membership (access group, authorization group, additional user data
group), maximum allowed sessions

EXAMPLE: User ppprv with pwd test, authorization group ppp-1, access
group ppp-1, billing group ppp-1 and maximum 1 session

\begin{tabular}{|c|c|c|c|c|c|}
\hline 
username&
password&
avpid&
accid&
bilid&
maxsess\tabularnewline
\hline
ppprv&
test&
test&
ppp-1&
ppp-1&
1\tabularnewline
\hline
\end{tabular}


\subsubsection{user access rights}

access group identifier; data in regular expression form about: permitted
phones, network devices ip, network devices ports; time brackets

can be multiply entries in one group

- time brackets format if time not set - it means any time time format: 

\begin{lyxcode}
{*}~{*}~{*}~{*}~{*}

\textasciicircum{}~\textasciicircum{}~\textasciicircum{}~\textasciicircum{}~\textasciicircum{}

|~|~|~|~+-~day~of~week~(0~-~6)

|~|~|~+-{}-{}-~month~of~year~(1~-~12)

|~|~+-{}-{}-{}-{}-~day~of~month~(1~-~31)

|~+-{}-{}-{}-{}-{}-{}-~hour~of~day~(0~-~24)

+-{}-{}-{}-{}-{}-{}-{}-{}-~minute~of~hour~(0~-~60)~
\end{lyxcode}
examples: 

{*} {*} {*} {*} 0,6 - access only on Sat and Sun 

{*} 20-8 {*} {*} 0,6 - access only on Sat and Sun from 20 to 8 

10-20,40-50 {*} {*} {*} 1-3,0,6 - more complex example.

Access from phone 322322 from any nas and any port

\begin{tabular}{|c|c|c|c|}
\hline 
phone&
nas&
port&
time\tabularnewline
\hline
d+322322\textbackslash{}/d+&
.{*}&
.{*}&
\tabularnewline
\hline
\end{tabular}

Access from any phone from nas 192.168.1.5 to any port

\begin{tabular}{|c|c|c|c|}
\hline 
phone&
nas&
port&
time\tabularnewline
\hline
.{*}&
\textasciicircum{}192.168.1.5\$&
.{*}&
\tabularnewline
\hline
\end{tabular}

Access from any phone, any nas and from Asy1

\begin{lyxcode}
phone~|~nas~|~~~port~~~|~time~-{}-{}-{}-{}-{}-+-{}-{}-{}-{}-+-{}-{}-{}-{}-{}-{}-{}-{}-{}-+-{}-{}-{}-{}-{}-{}-{}-{}-

~~.{*}~~|~~.{*}~|~\textasciicircum{}Asy1\$~~~|

-{}-{}-{}-{}-{}-{}-{}-{}-{}-{}-{}-{}-{}-{}-{}-{}-{}-{}-{}-{}-{}-{}-{}-{}-{}-{}-{}-{}-{}-{}-{}-{}-~
\end{lyxcode}
Portnames: Asy1 - AsyX for async dial-up connection (not-ISDN) for
ISDN portnames start with Se (Se0:10, ...) for example on my AS5300
i see (Asy1 - Asy240 and Se0:1-Se7:30,...) 

Access from any phone from nas 192.168.2.10 from Async1 and from any
nas from Async3 

\begin{lyxcode}
phone~|~~~~~~~~nas~~~~~|~~port~~|~time~-{}-{}-{}-{}-{}-+-{}-{}-{}-{}-{}-{}-{}-{}-{}-{}-{}-{}-{}-{}-{}-+-{}-{}-{}-{}-{}-{}-{}-+-{}-{}-{}-{}-{}-{}-{}-{}-

~.{*}~~~|~\textasciicircum{}192.168.2.10\$~|~\textasciicircum{}Asy1\$~|

~.{*}~~~|~~~~~~.{*}~~~~~~~~|~\textasciicircum{}Asy3\$~|~-{}-{}-{}-{}-{}-{}-{}-{}-{}-{}-{}-{}-{}-{}-{}-{}-{}-{}-{}-{}-{}-{}-{}-{}-{}-{}-{}-{}-{}-{}-{}-{}-{}-{}-{}-{}-{}-{}-{}-{}-{}-~
\end{lyxcode}
P.S. you should remember, that ports for PPP access can be named as
AsyX or as ttyX (if user use chap/pap, then AsyX, if it use access
from terminal or manual mode - ttyX)

if you wish create several usernames for several destination phone
numbers you can use target masks on phone. for example, i have two
modem pools - one on phone 320101, other on 8200, and i use next masks:
for 8-200: \textasciicircum{}8d+\textbackslash{}/421232\textbackslash{}:d+
for 32-01-01: \textasciicircum{}8d+\textbackslash{}/4212320101\$

you should know, that if you use analog access server like cisco 2511
or so on, then phone will be \char`\"{}async/\char`\"{}

also you should know the phone, which your NAS get from E1, you can
see it on Cisco equipment with command: debug isdn q931 (if you have
E1 connection to telephone switch) 

\begin{lyxcode}
access~for~voip~users~from~voice~gateway

phone~|~~~~~~nas~~~|~~~~~~port~~~~~~|~time~-{}-{}-{}-{}-{}-+-{}-{}-{}-{}-{}-{}-{}-{}-{}-{}-{}-+-{}-{}-{}-{}-{}-{}-{}-{}-{}-{}-{}-{}-{}-{}-{}-+-{}-{}-{}-{}-{}-{}-{}-{}-

~~.{*}~~|~\textasciicircum{}10.1.1.1\$~|~\textasciicircum{}FXO.{*}|\textasciicircum{}ISDN.{*}~|~-{}-{}-{}-{}-{}-{}-{}-{}-{}-{}-{}-{}-{}-{}-{}-{}-{}-{}-{}-{}-{}-{}-{}-{}-{}-{}-{}-{}-{}-{}-{}-{}-{}-{}-{}-{}-{}-{}-{}-{}-{}-{}-{}-{}-{}-
\end{lyxcode}

\subsubsection{user DEFAULT for LL access}

you should add user DEFAULT if your NAS doesn't support separate authorization
on several ports, if so - no need in this user 


\subsubsection{Additional User Data}

additional user data group identifier; authentication source identifier
(for use other than database source); enable password (this is for
network device)

authentication sources: only supported one - \char`\"{}db\char`\"{}

next possibly supported authensource (in future): \char`\"{}ccd\char`\"{}
or \char`\"{}ccf\char`\"{} - this is for CryptoCard

CryptoCard for authentication (thanks to Bradley Filmer and Alec Peterson)
!NOTE! CryptoCard support will be implemented in future versions

tacppd use file with name \char`\"{}etc/CRYPTOCard\char`\"{} or database
table with name \char`\"{}CRYPTOCardTokens\char`\"{}

{*}{*}{*} CryptoCard file {*}{*}{*}

cryptocard file looks like 

\begin{lyxcode}
\#~This~is~the~CRYPTOCard~authentication~definition~database~file.

\#

\#~Comments~start~with~a~\#~sign.~\#~Under~the~display~column:

\#~~~~~~~0~implies~hexadecimal~with~no~phone~display

\#~~~~~~~1~implies~hexadecimal~with~phone~display

\#~~~~~~~2~implies~decimal~with~no~phone~display

\#~~~~~~~3~implies~decimal~with~phone~display

\#

\#User~~~~Key(Encrypted)~~~~~~~~~~~~~~~Display~Challenge

\#-{}-{}-{}-{}-{}-{}-{}-{}-{}-{}-{}-{}-{}-{}-{}-{}-{}-{}-{}-{}-{}-{}-{}-{}-{}-{}-{}-{}-{}-{}-{}-{}-{}-{}-{}-{}-{}-{}-{}-{}-{}-{}-{}-{}-{}-{}-{}-{}-{}-{}-{}-{}-{}-{}-{}-{}-{}-{}-

TokUse~~a8b5c09e4ea6b503a65c7716383b67ff71406e5c7ec9a7b5~~~~3~~15952106

bbundy~~d0b669a9ea7892f4f8a40dedffd12bf34340aee363c8ef1c~~~~2~ajblog~~92b19a0cad6cc908e3e54e34efcf9802a54e15a2017d5066~~~~2~gbusha~~67d5af15d362b92ac30c87f3298aee2d5ecef59beb00723c~~~~2~edst~~~~6d78607d069a2fd3d8767087ba7bc5398be336c9860e038f~~~~2~
\end{lyxcode}
{*}{*}{*} CryptoCard database {*}{*}{*}

in common example standard cryptocard table looks like 

\begin{lyxcode}
CREATE~TABLE~CRYPTOCardTokens~(

~~~~~~~~UserID~char(64)~NOT~NULL,

~~~~~~~~DisplayID~char(8),

~~~~~~~~SerialNumber~char(12),

~~~~~~~~InitPIN~char(8),

~~~~~~~~EncryptedKey~char(64),

~~~~~~~~NextChallenge~char(8),

~~~~~~~~ProgDate~char(16),

~~~~~~~~GroupID~char(20),

~~~~~~~~Options~char(24),

~~~~~~~~AuthenCount~int(11),

~~~~~~~~PRIMARY~KEY~(UserID)

);~

UserID~DisplayID~SerialNumber~InitPIN~~EncryptedKey~~~~~~~~~NextChallenge~~ProgDate~~~~GroupID~~~Options~~AuthenCount~roman~~roman~~~~~498024979~~~~~~4751~~~cad3040197b565b4f964bd0d70d64ae8f90299b89~e67c44b~~~36621554~~~~~01/15/2002~Admin-Root~110354101~~~~~0
\end{lyxcode}

\subsection{authorization}

check user rights after success access for some protocols and some
resources

tacpp\_v1\_avp - authorization information tacpp\_v1\_cmd - per-command
authorization


\subsubsection{authorization information}

authorization group identifier; service; protocol; attribute-value
pair

AV-pairs - authorization attributes in Tacacs+ any av-pair consist
from two fields: avp name field and avp data field (service=ppp, protocol=ip,
etc)

You can use follow av-pairs:

service= slip, ppp, arap, shell, tty-daemon, connection, system, firewall

protocol= lcp, ip, ipx, atalk, vines, lat, xremote, tn3270, telnet,
rlogin, pad, vpdn, ftp, http, deccp, osicp, h323, unknown

cmd= shell (exec) command. Must be if service=shell can be cmd=NULL

cmd-arg= argument to shell (exec) command can be multiple

acl= number connection access-list (service=shell cmd=NULL)

inacl= identifier for interface in access-list

outacl= identifier for interface out access-list

zonelist= numeric zonelist value (AppleTalk only)

addr= network address 

addr-pool= identifier of address-pool

routing= boolean (is routing information on interface) permit/deny
send routing updates throught this interface

route= route for this interface \textless{}dst\_addr\textgreater{}
\textless{}mask\textgreater{} {[}\textless{}routing\_addr\textgreater{}]

timeout= timer for connection (minutes), 0 - no timeout

idletime= idle-timeout for connection (minutes)

autocmd= auto-command to run (service=shell cmd=NULL)

noescape= boolean (service=shell cmd=NULL) 

nohangup= boolean (service=shell cmd=NULL)

priv\_lvl= privilege level

remote\_user= remoute userid (TAC\_PLUS\_AUTHEN\_METH\_RCMD)

remote\_host= remote host (TAC\_PLUS\_AUTHEN\_METH\_RCMD)

callback-dialstring= NULL or dialstring, NULL - for request from user

callback-line= line for callback

callback-rotary= rotary

nocallback-verify= do not require authentication after callback 

SOME EXTERNAL AVPAIRS (NOT IN RFC, BUT WITH CISCO SUPPORT) - very-very
useful.... inacl\#\textless{}n\textgreater{} setup multiline access-list
(\textless{}n\textgreater{}-row number) inacl\#1=permit ip any any
inacl\#2=deny igrp ...

outacl\#\textless{}n\textgreater{}

route\#\textless{}n\textgreater{} multiline route entries

rte-ftr-in\#\textless{}n\textgreater{} input access list definition
for routing updates on interface rte-ftr-in\#0=router igrp 60 rte-ftr-in\#1=permit
0.0.3.4 255.255.0.0 rte-ftr-in\#2=deny any

rte-ftr-out\#\textless{}n\textgreater{} output acl for routing update

sap\#\textless{}n\textgreater{} static saps

route\#\textless{}n\textgreater{} route table 

sap-fltr-in\#\textless{}n\textgreater{} input sap filter list sap-fltr-out\#\textless{}n\textgreater{}
output sap filter list

pool-def\#\textless{}n\textgreater{} address pool definition pool-def\#1=DIALUP
10.1.1.1 10.1.1.100 pool-def\#2=INTERNAL 192.168.0.1 192.168.0.100

VoIP AVP: h323-billing-model=0/1 (credit/postpaid or debit/prepaid)
h323-credit-time= h323-credit-amount=

as i understand RFC, you can add your own av-pairs if client understand
it (be careful, sometimes it can be wrong for some clients), we don't
do any control for av-pairs in database - be carefull! 


\subsubsection{tacppd authorization data modifyers}

for better AV handling we have system of AV modifyers. The modifyers
- interface to internal functions, SQL servers and external programs
to change values during request

- Internal functions: set AVP as - attribute=INT:function the function
currently available - addrpool=number this is request to internal
address pooling system Example: \char`\"{}addr=INT:addrpool=10\char`\"{}
(get address from address pool number 10) for setting address pools
see config file documentation

- SQL functions: attribute=SQL:sql function request string request
to SQL for data. Get only first row value Example: \char`\"{}timeout=SQL:GetTimeoutFunction(\$name)\char`\"{}
it will requested as \char`\"{}SELECT GetTimeoutFunction(PppUserName)\char`\"{}
(this is example for PostgreSQL, for Oracle it will be some differ),
if i not mistake: \char`\"{}SELECT GetTimeoutFunction(PppUserName)
FROM DUAL\char`\"{} (see Oracle documentation about calling PL/SQL
functions) (you will need in this information only if you will create
your own database driver)

External program: attribute=EXT:external program request string Example:
\char`\"{}timeout=EXT:get\_timeout.pl\char`\"{} or \char`\"{}timeout=EXT:get.pl
\$name\char`\"{} NOTE: see BUGS file for status of bug 20010716 (currently
this feature doesn't work) The programs can be placed ONLY into \char`\"{}ext\char`\"{}
subdirectory in tacppd tree. 

For sending information to external requests you can use additional
variables: \$name - during this request this will be changed to current
username string \$port - nas port name \$nas - network access server
\$phone - phone string if it is dialup connection

Example: \char`\"{}timeout=SQL:time\_table\_function('\$name','\$nas','\$phone')\char`\"{}

!NOTE! - external program modifyer very buggy and don't usable while


\subsubsection{common authorization axamples}

usually any ISP with DIAL-UP services has several authorization groups:

1. ordinary PPP with use NAS internal address pool service=ppp protocol=lcp
service=ppp protocol=ip addr-pool=DIALUP \# below only if users use
terminals for insert username/password \# (some non-standard, advanced
or stupid dial-up clients) service=shell cmd= service=shell autocmd=ppp
service=shell noescape=true

2. group = username - for ppp with static ip addr service=ppp protocol=lcp
service=ppp protocol=ip addr=194.85.113.100

3. ppp with use tacppd internal ip addr pooling system this is our
addition, but for NAS it will be converted to string \char`\"{}addr=address\_from\_pool\_system\char`\"{}
service=ppp protocol=lcp service=ppp protocol=ip addr=INT:addrpool=1

4. ppp with additional ip filters service=ppp protocol=lcp service=ppp
protocol=ip addr-pool=DIALUP service=ppp protocol=ip inacl\#1=deny
ip any 192.168.0.0 0.0.0.255 service=ppp protocol=ip inacl\#2=permit
ip any any

5. uucp (rlogin access to uucp server) service=shell cmd= service=shell
autocmd=rlogin aaa.bbb.ru /user uuuser service=shell noescape=true

6. admin access (unrestricted) service=shell cmd= 

7. async tunnel: service=shell cmd= service=shell autocmd=telnet 192.168.10.100
3162 /stre am service=shell noescape=true

some comments:

a) if you use ppp multilink (for example, you have ISDN users), you
should add: service=ppp protocol=multilink max-links=2

b) ppp callback service service=ppp protocol=lcp callback-dialstring=
service=shell callback-dialstring= service=shell nocallback-verify=1


\subsubsection{specific examples}

dial-in PPP with pool DIALUP

\begin{tabular}{|c|c|c|}
\hline 
service&
protocol&
av-pair\tabularnewline
\hline
\hline 
ppp&
lcp&
\tabularnewline
\hline 
ppp&
ip&
addr-pool=DIALUP\tabularnewline
\hline 
shell&
&
cmd=\tabularnewline
\hline 
shell&
&
autocmd=ppp\tabularnewline
\hline 
shell&
&
noescape=true\tabularnewline
\hline
\end{tabular}

dial-in PPP with pool DIALUP and for ISDN

\begin{lyxcode}
service~|~protocol~|~av-pair~-{}-{}-{}-{}-{}-{}-{}-+-{}-{}-{}-{}-{}-{}-{}-{}-{}-+-{}-{}-{}-{}-{}-{}-{}-{}-{}-{}-{}-{}-{}-{}-{}-{}-{}-

~~ppp~~~|~~~lcp~~~~|

~~ppp~~~|~~~ip~~~~~|~addr-pool=DIALUP

~~ppp~~~|multilink~|~max-links=2~-{}-{}-{}-{}-{}-{}-{}-{}-{}-{}-{}-{}-{}-{}-{}-{}-{}-{}-{}-{}-{}-{}-{}-{}-{}-{}-{}-{}-{}-{}-{}-{}-{}-{}-{}-{}-{}-~
\end{lyxcode}
dial-in PPP with static ip addr

\begin{lyxcode}
service~|~protocol~|~av-pair~-{}-{}-{}-{}-{}-{}-{}-+-{}-{}-{}-{}-{}-{}-{}-{}-{}-+-{}-{}-{}-{}-{}-{}-{}-{}-{}-{}-{}-{}-{}-{}-{}-{}-{}-

~~ppp~~~|~~~lcp~~~~|

~~ppp~~~|~~~ip~~~~~|~addr=10.1.1.10

~~shell~|~~~~~~~~~~|~cmd=

~~shell~|~~~~~~~~~~|~autocmd=ppp

~~shell~|~~~~~~~~~~|~noescape=true~-{}-{}-{}-{}-{}-{}-{}-{}-{}-{}-{}-{}-{}-{}-{}-{}-{}-{}-{}-{}-{}-{}-{}-{}-{}-{}-{}-{}-{}-{}-{}-{}-{}-{}-{}-{}-{}-~
\end{lyxcode}
async tunneling

\begin{lyxcode}
service~|~protocol~|~av-pair~-{}-{}-{}-{}-{}-{}-{}-+-{}-{}-{}-{}-{}-{}-{}-{}-{}-+-{}-{}-{}-{}-{}-{}-{}-{}-{}-{}-{}-{}-{}-{}-{}-{}-{}-{}-{}-{}-{}-{}-{}-{}-{}-{}-{}-{}-{}-{}-{}-{}-{}-{}-{}-{}-{}-~~~~~~shell~|~~~~~~~~~~|~cmd=

~~shell~|~~~~~~~~~~|~autocmd=telnet~192.168.1.112~/stream~~~~~~~shell~|~~~~~~~~~~|~noescape=true~-{}-{}-{}-{}-{}-{}-{}-{}-{}-{}-{}-{}-{}-{}-{}-{}-{}-{}-{}-{}-{}-{}-{}-{}-{}-{}-{}-{}-{}-{}-{}-{}-{}-{}-{}-{}-{}-{}-{}-{}-{}-{}-{}-{}-{}-{}-{}-{}-{}-{}-{}-{}-{}-{}-{}-{}-{}-
\end{lyxcode}
shell connect (all commands)

\begin{lyxcode}
service~|~protocol~|~av-pair~-{}-{}-{}-{}-{}-{}-{}-+-{}-{}-{}-{}-{}-{}-{}-{}-{}-+-{}-{}-{}-{}-{}-{}-{}-{}-{}-{}-{}-{}-{}-{}-{}-{}-{}-

~~shell~|~~~~~~~~~~|~cmd=~-{}-{}-{}-{}-{}-{}-{}-{}-{}-{}-{}-{}-{}-{}-{}-{}-{}-{}-{}-{}-{}-{}-{}-{}-{}-{}-{}-{}-{}-{}-{}-{}-{}-{}-{}-{}-{}-~
\end{lyxcode}
connect for UUCP (autocommand)

\begin{lyxcode}
service~|~protocol~|~av-pair~-{}-{}-{}-{}-{}-{}-{}-+-{}-{}-{}-{}-{}-{}-{}-{}-{}-+-{}-{}-{}-{}-{}-{}-{}-{}-{}-{}-{}-{}-{}-{}-{}-{}-{}-{}-{}-{}-{}-{}-{}-{}-{}-{}-{}-{}-{}-{}-{}-{}-{}-{}-{}-{}-{}-{}-{}-{}-~~shell~~|~~~~~~~~~~|~cmd=

~shell~~|~~~~~~~~~~|~autocmd=rlogin~smtp.aaa.ru~/user~uuuser~~~shell~~|~~~~~~~~~~|~noescape=true~-{}-{}-{}-{}-{}-{}-{}-{}-{}-{}-{}-{}-{}-{}-{}-{}-{}-{}-{}-{}-{}-{}-{}-{}-{}-{}-{}-{}-{}-{}-{}-{}-{}-{}-{}-{}-{}-{}-{}-{}-{}-{}-{}-{}-{}-{}-{}-{}-{}-{}-{}-{}-{}-{}-{}-{}-{}-{}-{}-{}-
\end{lyxcode}
for shell connect with restricted command set you should use PER-COMMAND
AUTHORIZATION 

set voice user parameter with voice gateway

\begin{lyxcode}
~~service~~~|~protocol~|~av-pair~-{}-{}-{}-{}-{}-{}-{}-{}-{}-{}-{}-+-{}-{}-{}-{}-{}-{}-{}-{}-{}-+-{}-{}-{}-{}-{}-{}-{}-{}-{}-{}-{}-{}-{}-{}-{}-{}-{}-{}-{}-{}-{}-

~connection~|~~~h323~~~|~~h323-credit-time=120~-{}-{}-{}-{}-{}-{}-{}-{}-{}-{}-{}-{}-{}-{}-{}-{}-{}-{}-{}-{}-{}-{}-{}-{}-{}-{}-{}-{}-{}-{}-{}-{}-{}-{}-{}-{}-{}-{}-{}-{}-{}-{}-{}-{}-{}-
\end{lyxcode}

\subsubsection{per-command authorization}

description: per-command authorization for cmd= and cmd-arg= av-pairs

authorization group id; regex with permitted commands; regex with
deny commands; regex with permitted command arguments; regex with
deny command arguments

.{*} - all {[}\textasciicircum{}.{*}] - nothing

permit any commands and arguments 

\begin{lyxcode}
~cmdperm~|~cmddeny~|~argperm~|~argdeny~-{}-{}-{}-{}-{}-{}-{}-{}-+-{}-{}-{}-{}-{}-{}-{}-{}-+-{}-{}-{}-{}-{}-{}-{}-{}-+-{}-{}-{}-{}-{}-{}-{}-{}-

~~~.{*}~~~~|~~~~~~~~~|~~~~.{*}~~~|~-{}-{}-{}-{}-{}-{}-{}-{}-+-{}-{}-{}-{}-{}-{}-{}-{}-+-{}-{}-{}-{}-{}-{}-{}-{}-+-{}-{}-{}-{}-{}-{}-{}-{}-
\end{lyxcode}
deny any commands and arguments

\begin{lyxcode}
~cmdperm~|~cmddeny~|~argperm~|~argdeny~-{}-{}-{}-{}-{}-{}-{}-{}-+-{}-{}-{}-{}-{}-{}-{}-{}-+-{}-{}-{}-{}-{}-{}-{}-{}-+-{}-{}-{}-{}-{}-{}-{}-{}-

~~~~~~~~~|~~~.{*}~~~~|~~~~~~~~~|~~~.{*}~-{}-{}-{}-{}-{}-{}-{}-{}-+-{}-{}-{}-{}-{}-{}-{}-{}-+-{}-{}-{}-{}-{}-{}-{}-{}-+-{}-{}-{}-{}-{}-{}-{}-{}-
\end{lyxcode}

\subsection{accounting}

logging activities

tacpp\_v1\_log

description: log entries table (only stop entries) server char(16)
-- server -- servtime datetime -- time in server -- start\_time datetime
-- nas internal start time -- task\_id int -- nas task\_id -- username
char(32) -- username -- nas char(16) -- NAS -- 

port char(10) -- nas port -- service char(10) -- service -- cfrom
char(64) -- phone or ip addr -- ip char(16) -- ip addr of client --
protocol char(10) -- protocol -- elapsed int -- elapsed time -- bytes\_in
int -- bytes in port -- bytes\_out int -- bytes out from port -- disc\_cause
int -- disconnect cause -- disc\_cause\_ext int -- ext disconnect
cause -- rx\_speed int -- connect read speed -- tx\_speed int 

-- connect write speed -- h323\_call\_origin char(10) -- answer/originate/
-- h323\_call\_type char(15) -- Telephony/ -- h323\_disconnect\_cause
int -- -- h323\_voice\_quality int -- -- h323\_connect\_time datetime
-- -- h323\_disconnect\_time datetime -- -- h323\_remote\_address
char(16) -- remot e ip addr --

the tacppd store accounting information in two destinations: 1. text
file 2. database it stores all av-pairs wich it receives from NAS


\subsection{traffic information}

server can store ifInOctets/ifOutOctets information from NAS and routers.
This information collects from device during SNMP polling process.

description: traffic log via interface counters server char(16) --
the server IP -- curtime datetime -- time of entry -- port char(16)
-- info for port -- nas char(16) -- on nas ip -- bytein float -- ifInOctets
-- byteout float -- ifOutOctets --


\subsection{billing data}

tacpp\_v1\_bil tacpp\_v1\_bilres

description: billing resource table description bilid char(32) --
billing group -- bilidname char(50) -- billing group name -- bilmod
char(32) -- resource module name --

description: billing resource table bilid char(32) -- billing group
-- attribute char(32) -- attribute name -- value char(32) -- attribute
val ue --

!NOTE! - billing system no available, it will be in future releases


\subsection{additional data notes}

The VoIP session (on FXO) logs 4 entries for all connections: 1. start
from port 2. start from user connection 3. stop from user connection
4. stop from port

in 1 and 4 you can see FXO port number also, opening ports doing authentication
with username of your pots peer number mask (3640 with FXO for example).
But it doesn't care about result of it.

in new IOS versions i see some changes in this.

The VoIP ports on Cisco 36xx series with NM-HDV voice modules generate
port names as 'ISDN number:number' unless you use \char`\"{}aaa nas
port voip\char`\"{} configuration command - than portname is 'Serial
num/num:num ' 

if you use tacacs+ for Cisco VoIP gateway authentication, than you
should write your TCL script so, that it should be aware of: 1. in
first authentication session voip gw does not send information to
tacacs+ about source/destination phone numbers 2. it does not send
information about phone numbers in authorization request packets,
and it does not send any av-pair in authorization start 3. authorization
request from TCL script create two requests to tacacs+ server - authentication
with setted username/pin and with source/destination phone numbers
and after that it generate void authorization request, but for send
values from tacacs+ to TCL you should use authorization responce in
form: h323-credit-time=digit (or any other h323-related av-pairs,
see documentation upper or in Cisco documentation)

Due to logging information where i plan to use device DNS names, not
ip addresse s (due to more right logfile understanding, and for statistics),
carefully check your DNS entries about network device and database
servers names in native and r everse DNS zones. May be the best way
- use /etc/hosts file for it. 


\section{TCI (tacppd control interface)}

For use external programs for control active users and see devices
availability. It uses TCP connection with simple commands set. You
can use telnet or any simple TCP connection from your protgrams for
access to TCI. For use TCI you should configura listener and TCI users
(2 level access).


\subsection{TCI listener}

\begin{lyxcode}
listener~11001~

~type~tci

~maxconnect~10~

~acl~3~

~exit~
\end{lyxcode}

\subsection{TCI users}

\begin{lyxcode}
manager~tciadm~

~password~'tcipwd'~

~type~tci~

~acl~1~

~exit~
\end{lyxcode}
TCI users have access to all TCI commands..


\subsection{TCI commands}

The TCI session require authentication, after connect open you have
to enter username, and password, after that you can use TCI commands
for see list of active users on network devices, see list of network
devices and it's status, drop user from all devices, clear port on
specifyed device. All output from tacppd will have form: {}``\#:\textbackslash{}tresponse''.
But \textbackslash{}t this is TAB (\textbackslash{}t). I will write
response in form {}``\#: Ok'', but you should know, that space after
{}``\#:'' - this is TABulation. Also every responce will ended with
symbols \textbackslash{}r\textbackslash{}n, but i will not write it
in examples. In every session you can send many commands. Also you
can have TCI session opened for a long time.


\subsubsection{authentication}

\begin{lyxcode}
s06:~\{7\}~\%~telnet~localhost~11001

Trying~127.0.0.1...~Connected~to~localhost.~

Escape~character~is~'\textasciicircum{}]'.~

tciadm~

tcipwd~

\#:~Ok
\end{lyxcode}
If you use wrong username/password, message will be {}``\#: Error''


\subsubsection{{}``showdevices'' command}

You should send command {}``showdevices'' and you will receive device
names, ips and status Up/Down in next form.

\begin{lyxcode}
showdevices~

\#:~Begin

vgw~192.168.113.21~Up~

vgk~192.168.113.15~Up~

as01~192.168.113.28~Up~

as02~192.168.113.12~Up~

\#:~End
\end{lyxcode}
Output about devices always inside Begin/End. Spaces in output - this
is TAB (\textbackslash{}t) symbols. Every string ended by {}``\textbackslash{}r\textbackslash{}n''.


\subsubsection{{}``showusers'' command}

You should send command {}``showusers'' and you will receive device
with ip and name and users on this device with username, port, phone\_from/phone\_to
and ip address.

\begin{lyxcode}
showusers~

\#:~Begin

\#:~Device~192.168.113.21(vgw)~

501~Serial1/0:31~74212322311/74212323794~0.0.0.0~

\#:~Device~192.168.113.15(vgk)~

\#:~Device~192.168.113.28(as01)~

\#:~Device~192.168.113.12(as02)

\#:~End
\end{lyxcode}
All spaces here - this is TAB (\textbackslash{}t) symbol. Every string
ended by {}``\textbackslash{}r\textbackslash{}n''.


\subsubsection{{}``checkpwd'' command}

For check crypted password.


\subsubsection{{}``cryptpwd'' command}

For crypt password in tacppd style.


\subsubsection{{}``clearuser'' command}

You sould send command {}``clearuser'' after that send username
for clear. Server returns reply {}``Ok''.

\begin{lyxcode}
clearuser

pppuser1100

\#:~Ok
\end{lyxcode}
All spaces here - this is TAB (\textbackslash{}t) symbol. Every string
ended by {}``\textbackslash{}r\textbackslash{}n''.


\subsubsection{cleardeviceport command}

You should send command {}``cleardeviceport'' after that send device
ip and port which will dropped. Server returns {}``Ok''.

\begin{lyxcode}
cleardeviceport

192.168.113.28

Async21

\#:~Ok
\end{lyxcode}
All spaces here - this is TAB (\textbackslash{}t) symbol. Every string
ended by {}``\textbackslash{}r\textbackslash{}n''.


\subsubsection{exit command}

For end tci session, use command {}``exit'' or {}``quit''.


\section{Bundle setup and configure}

Bundle use TIP (tacppd interchange protocol) - this is UDP packets
with multicast addresses. Each tacppd server can send and receive
this packets.


\subsection{Bundle operation}

There are 4 types of packets for bundle operation: device track packet,
user track packet, keepalive packet and start packet. Device track
packet has information about device IP and it's status: is device
Up or Down now. User track packet has information about currently
active users and some information about it. Keepalive packet - about
tacppd server activity with it's priority information. Start packet
- notification about new member in bundle. Tacppd sends device track
packet if it done icmp or snmp polling and detects device status change.
The device track packet with information about down device can be
sent only as result of polling operation. User track packet can be
sent as result of snmp polling operation with user status Up or Down,
and in case of receive tacacs+ accounting information, and can set
user status in Up or Down. Keepalive packets - for selection of active
server - this is server which do polling operation and do users drops
by snmp - and provide information about server priority and do information
about any server availability. Also server priority value have device
track and user track packets and it replace keepalive packet if presents.
Keepalive packets send every tacppd every 10 seconds. It has tacppd
server ip and it's priority.


\subsection{Active tacppd server}

In bundle one tacppd will be active. Active status means, that this
tacppd have to do snmp/icmp network devices polling, and also this
tacppd will send to network devices snmp commands for drop users.
Also this tacppd resource system will do updates to database during
resources changes. This is tacppd with highest bundle priority value.
The algorythm of active tacppd selection next: after start every tacppd
matched as inactive, send start packet and wait 30 sec. If it receive
keepalive, start, dt or ut packet in bundle, than it store information
about this tacppd, and compare priority values with own priority value.
The maximum priority tacppd set as active. If localhost tacppd priority
higher, this server - active. In other case - inactive. If any receive
start packet, than it should send set of devicetrack and usertack
packets. The list of tacppd servers has expire value of 61 sec, if
during this time no information come from any tacppd, than this tacppd
server delete from list and if it was active, than do new active server
select. Any tacppd send keepalives every 10 secs in case, if no dt
or ut packets were created during this period. If it were, than it
wait 10 secafter own last dt or ut packet was sent.


\subsection{Multicast configuration}


\subsubsection{Linux}

If you use tacppd bundle, you must enable multicast on your Linux
box. Exellent documentation available at Multicast-HOWTO (http://www.linux.org/docs/ldp/howto/Multicast-HOWTO.html).
In several words:

\begin{description}
\item [{Check}] if it is enabled, and if not, enable Multicasting in the
kernel 
\item [{Configure}] multicast routing with command:
\end{description}
/sbin/route add -net 224.0.0.0 netmask 240.0.0.0 dev eth0

\begin{description}
\item [{See,}] if Linux box connected to multicast group, see /proc/net/igmp
file. You will see something like:
\end{description}
\begin{lyxcode}
Idx~Device~:~Count~Querier~Group~Users~Timer~Reporter~

1~lo~:~0~V2~010000E0~1~0:ED688817~0~

2~eth0~:~2~V2~010101EF~1~0:FFFFED3F~1~

~~~~~~~~~~~~~~010000E0~1~0:ED688817~0
\end{lyxcode}
(if you stop tacppd, than you do not see string where Reporter=1).


\subsubsection{FreeBSD}


\subsubsection{SUN Solaris}


\subsubsection{All platforms}

Also try to do ping to a multicast group address. The bundle configuration
in tacppd.conf file should be looking like:

\begin{lyxcode}
access~4~

~permit~\textasciicircum{}239.1.1.1\$~

exit

listener~11000~

~type~bundle~

~maxconnect~10~

~acl~4~

exit

bundle~239.1.1.1:11000~

~description~'tacppd~multicast~group'~

~cryptokey~key1~

~ttl~1~

exit
\end{lyxcode}
The ttl=1 only if all nodes connected to one LAN. If you have routers
and other network between nodes, set ttl to more. Also set routers
for multicast routing.


\subsubsection{Multicast routing configuration}

If you have use tacppds installed in several servers situated in several
networks and want join tacppds to bundle, you have to setup multicast
routing. There are a lot of documentation on http://www.cisco.com
about multicast routing configuration. We are see only several simple
examples. For example we have two Ethernet segments which interconnected
by Catalyst switch in different vlans, interconnected by Cisco router.

Router configuration:

\begin{lyxcode}
!~enable~multicast~routing

ip~multicast-routing



!~first~subinterface~configuration

interface~FastEthernet0/0.2

~description~First~TACPPD~VLAN

~encapsulation~dot1Q~2

~ip~pim~sparse-mode

~ip~cgmp



!~second~subinterface~configuration

interface~FastEthernet0/0.3

~description~Second~TACPPD~VLAN

~encapsulation~dot1Q~3

~ip~pim~sparse-mode

~ip~cgmp



ip~pim~rp-address~your.router.ip.address
\end{lyxcode}
In this example we are using PIM sparse-mode, and RP of our router.
Also we set CGMP enabled on subinterfaces, but instead you can use
IGMP Snooping feature on the Catalysts, but i prefer CGMP, i think,
that it looks like better.

Catalyst switch configuration:

\begin{lyxcode}
!~disable~igmp~snooping~feature~(cgmp~enabled~by~default)

no~ip~igmp~snooping



!~trunk~ethernet~interface

interface~FastEthernet~0/1

~description~Router~trunk~connect

~switchport~mode~trunk



!~first~ethernet~interface

interface~FastEthernet0/5

~description~First~TACPPD~server~host

~switchport~access~vlan~2



!~second~ethernet~interface

interface~FastEthernet0/6

~description~Second~TACPPD~server~host

~switchport~access~vlan~3
\end{lyxcode}
In this example we disable IGMP snooping feature and set Catalyst
interfaces.

Remember, that you have to use multicast TTL values more than 1 if
you will use multicast routing, every hop require add 1 to TTL. For
our example set TTL to 2.


\section{Peers setup and configure}


\section{Tacppd using}


\subsection{Stand-alone configuration without nework devices (for testing)}

\begin{lyxcode}
!~tacppd~configuration~file,~build~by~admin~Thu~Jun~20~09:06:37~2002~

config



~!~debug

~debug~file

~~facility~db

~~facility~pool

~~facility~billing

~~facility~track

~~facility~tacacs

~~facility~aaa

~~facility~snmp

~~exit

~debug~syslog

~~exit







~!~access~list

~access~1

~~permit~\textasciicircum{}127.0.0

~~permit~\textasciicircum{}192.168.0

~~exit

~access~2

~~permit~\textasciicircum{}127.0.0

~~permit~\textasciicircum{}192.168.0

~~exit

~access~3

~~permit~\textasciicircum{}127.0.0

~~permit~\textasciicircum{}192.168.0

~~exit

~access~4

~~permit~\textasciicircum{}239.1.1

~~exit



~!~system~manager

~manager~admin

~~password~'admin-pwd'

~~type~terminal

~~acl~1

~~exit

~manager~webmin

~~password~'webmin-pwd'

~~type~http

~~acl~1

~~exit



~!~tcp/udp~listener

~listener~2222

~~type~terminal

~~maxconnect~2

~~acl~2

~~exit

~listener~10000

~~type~tacacs

~~maxconnect~100

~~acl~3

~~exit

~listener~8888

~~type~http

~~maxconnect~10

~~acl~2

~~exit

~listener~12001

~~type~netflow

~~maxconnect~10

~~acl~2

~~exit

~listener~11000

~~type~bundle

~~maxconnect~10

~~acl~4

~~exit



~!~address~pool

~pool~1

~~addr~10.1.1.1

~~addr~10.1.1.2

~~addr~10.1.1.3

~~exit



~!~network~access~server

~device~192.168.0.2

~~description~'test'

~~tacacskey~key

~~snmpcommunity~'common'

~~loginstring~'login++:'

~~pwdstring~'pwd++:'

~~defauthorization~no

~~module~cisco.so

~~polldelay~5

~~inttrfcount~no

~~snmppolling~no

~~icmppolling~yes~

~~shutdown~no

~~exit

~device~127.0.0.1

~~description~'our'

~~tacacskey~key

~~snmpcommunity~'public'

~~loginstring~'login++:'

~~pwdstring~'passw++:'

~~defauthorization~no

~~module~cisco.so

~~polldelay~60

~~inttrfcount~no

~~snmppolling~no

~~icmppolling~yes

~~shutdown~no

~~exit



~!~database

~database~tacppd-db@192.168.0.2:5432

~~cryptokey~'dbkey'

~~login~tacppd

~~password~tacppd-pwd

~~module~pgsql.so

~~shutdown~no

~~exit

~database~tacppd-db@192.168.0.3:5432

~~cryptokey~'dbkey'

~~login~tacppd

~~password~tacppd-pwd

~~module~pgsql.so

~~shutdown~no

~~exit



~!~tacppd~bundle

~bundle~239.1.1.1:11000

~~cryptokey~key1

~~priority~10

~~ttl~1

~~exit



~!~billing~module

~billing~0

~~description~'void'

~~module~none.so

~~shutdown~yes

~~exit



~!~rouming~peer

~peer~www1.kht.ru:8001

~~description~'none'

~~cryptokey~key1

~~timeshift~+2

~~mask~\textasciicircum{}Pppwww1.{*}

~~mask~\textasciicircum{}ppp.{*}

~~exit

~peer~www2.kht.ru:8002

~~description~'none'

~~cryptokey~key2

~~mask~\textasciicircum{}Pppwww2.{*}

~~mask~\textasciicircum{}www.{*}

~~exit~



exit
\end{lyxcode}
You can use LibTACPLUS (http://sourceforge.net/projects/libtacplus)
for emulate network device (in libtacplus's directory /samples you
will find simple tacacs+ client).


\subsection{Configuration}


\section{NetFlow collector}

Tacppd has inbuilt NetFlow information collector. NetFlow - this is
Cisco(R) technology, used in Cisco devices for provide information
about ip traffic. Router generate UDP packets with information about
data flows. Server can accumulate it and do accounting and billing.
The most problem - information volume. Usually we cant directly log
this NetFlow packets - this is too large logfiles, and too much resources
required for process it. Tacppd has simple aggregation system, where
it collects flow information in memory for 15 minutes and writes to
database only aggregated information. In future versions we plan add
some mechanism for provide detailed logs for configured ip addresses.
But currently we write only aggregated data. It greatly reduce data
volume and do possible billing and normal accounting of traffic information
on per-ip basis.

see tacpp.conf.orig or use CLI for configure NetFlow collector. The
log file name with NetFlow information looks like nf20011116.log and
recreate every day

log file format differ in depend of NetFlow packet versions 

\begin{lyxcode}
format~v1:

ip\_addr~~~~~~~~~IP~addr~of~router~which~sends~NetFlow~information

version~~~~~~~~~version~of~NetFlow~packets

SysUptime~~~~~~~time~in~msecs~since~router~booted

unix\_secs~~~~~~~current~time~in~seconds~since~0000~UTC~1970~unix\_nsecs~~~~~~residual~nanoseconds~since~0000~UTC~1970~First~~~~~~~~~~~SysUptime~at~start~of~flow~Last~~~~~~~~~~~~SysUptime~of~last~packet~of~the~flow

protocol~~~~~~~~IP~protocol,~6-TCP,~17-UDP,~1-ICMP

srcaddr~~~~~~~~~ip~source~addr

srcport~~~~~~~~~TCP/UDP~source~port~number

dstaddr~~~~~~~~~ip~destination~addr

dstport~~~~~~~~~TCP/UDP~destination~port

nexthop~~~~~~~~~next~hop~router's~IP~addr

input~~~~~~~~~~~input~interface~index

output~~~~~~~~~~output~interface~index

dPkts~~~~~~~~~~~packets~sent~in~time~between~First~and~Last~dOctets~~~~~~~~~octets~sent~in~time~between~First~and~Last



format~v5:~

ip\_addr~~~~~~~~~IP~addr~of~router~which~sends~NetFlow~information

version~~~~~~~~~version~of~NetFlow~packets

SysUptime~~~~~~~time~in~msecs~since~router~booted

unix\_secs~~~~~~~current~time~in~seconds~since~0000~UTC~1970~unix\_nsecs~~~~~~residual~nanoseconds~since~0000~UTC~1970~flow\_sequence~~~sequence~number~of~total~flows~seen~engine\_type~~~~~type~of~flow~switching~engine~(RP,VIP,etc)~engine\_id~~~~~~~slot~number~of~flow~switching~engine~First~~~~~~~~~~~SysUptime~at~start~of~flow~Last~~~~~~~~~~~~SysUptime~of~last~packet~of~the~flow~protocol~~~~~~~~IP~protocol,~6-TCP,~17-UDP,~1-ICMP~srcaddr~~~~~~~~~ip~source~addr~srcport~~~~~~~~~TCP/UDP~source~port~number~dstaddr~~~~~~~~~ip~destination~addr~dstport~~~~~~~~~TCP/UDP~destination~port~nexthop~~~~~~~~~next~hop~router's~IP~addr~input~~~~~~~~~~~input~interface~index~output~~~~~~~~~~output~interface~index~dPkts~~~~~~~~~~~packets~sent~in~time~between~First~and~Last~dOctets~~~~~~~~~octets~sent~in~time~between~First~and~Last~src\_as~~~~~~~~~~source~peer/origin~Autonomous~System~dst\_as~~~~~~~~~~destination~peer/origin~Autonomous~System~src\_mask~~~~~~~~source~route's~mask~bits~

dst\_mask~~~~~~~~destination~route's~mask~bits~tos~~~~~~~~~~~~~IP~Type-of-Service
\end{lyxcode}
end


\section{TACPPD logfiles}

Tacppd use several logfiles for provide information with several levels.
Also it can use database tables for store some logging information.
Every logfile entry starts with fragment with data and time, divided
from other part of string by brackets {[}]. In common way it looks
like {[}2003/02/18-03-23:10:56] when you can see year/month/day-day\_of\_week-hour:minute:second.
All data gets from localhost clock, and writes in local timezone.
In later releases possible deprecate field with day of week value.


\subsection{tacacs+ log}

Tacacs+ accounting information writes in file YYYYMM.log.

Example:

\begin{lyxcode}
{[}2002/11/06-04-17:15:28]~action=START~user=84212322311~port=Serial1/0:21~~from=84212322311~nas=192.168.113.21~task\_id=27~start\_time=1036566928~timezone=VSK~service=connection~protocol=h323~

{[}2002/11/06-04-17:15:54]~action=STOP~user=111~port=Serial1/0:21~~from=84212322311~nas=192.168.113.21~task\_id=27~start\_time=1036566928~~timezone=VSK~service=connection~protocol=h323~~h323-gw-id=vgw.tts.khb.cisco~h323-incoming-conf-id=5CBC9187~F08E11D6~8058C840~5D76F8B~3~h323-call-origin=answer~h323-call-type=Telephony~~h323-setup-time=17:15:28.589~VSK~Wed~Nov~6~2002~subscriber=RegularLine~~in-portgrp-id=\#\#\#~320320~\#\#\#~h323-connect-time=17:15:28.599~VSK~Wed~Nov~6~2002~~h323-disconnect-time=17:15:53.933~VSK~Wed~Nov~6~2002~h323-disconnect-cause=10~~tariff-type=Unknown~h323-voice-quality=0~pre-bytes-in=0~pre-bytes-out=0~~pre-paks-in=0~pre-paks-out=0~bytes\_in=0~bytes\_out=64160~paks\_in=0~paks\_out=401~connect-progress=101~elapsed\_time=25~nas-rx-speed=0~nas-tx-speed=0
\end{lyxcode}
this is full output of TACACS+ accounting packet.


\subsection{tacacs+ database log}

For billing we should have more useful log for provide acoounting
and export information to other software.


\subsection{netflow information log}


\subsection{netflow database log}

For provide per-traffic tariffs we should have netflow information
available via database.


\subsection{events log}

Events in tacppd

Example:

\begin{lyxcode}
{[}2003/02/15-07-19:25:29]~create~Database~Module~lpmx@192.168.113.247:5432~{[}2003/02/15-07-19:25:29]~-~tcp-servers~start~-~

{[}2003/02/15-07-19:25:29]~Server~terminal~now~listen~at~port~2222~{[}2003/02/15-07-19:25:30]~Server~tacacs~now~listen~at~port~10000~{[}2003/02/15-07-19:25:31]~Server~bundle~now~listen~at~port~11000~{[}2003/02/15-07-19:25:32]~Server~tci~now~listen~at~port~11001~{[}2003/02/15-07-19:25:33]~Server~netflow~now~listen~at~port~12001~{[}2003/02/15-07-19:25:45]~NAS~192.168.113.21~(vgw)~go~Up~

{[}2003/02/15-07-19:25:45]~NAS~192.168.113.15~(vgk)~go~Up~

{[}2003/02/15-07-19:25:49]~NAS~192.168.113.28~(as01)~go~Up~

{[}2003/02/15-07-19:25:55]~NAS~192.168.113.12~(as02)~go~Up~
\end{lyxcode}

\subsection{info log}


\subsection{errors log}


\subsection{debug log}

Several debug information as configured in config.

Example:

\begin{lyxcode}
{[}2003/02/15-07-15:13:54]~TACACS:~create~new~tacacs+~session~with~192.168.113.21~sessid=7029305

{[}2003/02/15-07-15:13:54]~TACACS:~Session~Process~type=1~seq=1~{[}2003/02/15-07-15:13:54]~TACACS:~start->action=1~start->service=10~start->authen~\_type=1~version=192

{[}2003/02/15-07-15:13:54]~TACACS:~Proccess~ASCII~login/password~from~192.168.113.21~for~111

{[}2003/02/15-07-15:13:54]~TACACS:~try~nopassword~db\_authentication~{[}2003/02/15-07-15:13:54]~TRACK:~add~inactive~user~111/Serial1/0:30~to~list~{[}2003/02/15-07-15:13:54]~TACACS:~Session~Process~type=1~seq=3~{[}2003/02/15-07-15:13:54]~AAA:~~-~check~by~rem\_addr(phone)~(.{*}~with~)~success~
\end{lyxcode}

\section{Database logging}

Tacppd also writes some logs directly to database tables. This is
tacacs+ accounting, NetFlow data and some other. This information
used by contributed with tacppd billing/accounting system.
\end{document}
