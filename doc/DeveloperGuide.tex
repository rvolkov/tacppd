%% LyX 1.4.1 created this file.  For more info, see http://www.lyx.org/.
%% Do not edit unless you really know what you are doing.
\documentclass[english]{article}
\usepackage[]{fontenc}
\usepackage[latin1]{inputenc}
\usepackage{graphicx}

\makeatletter

%%%%%%%%%%%%%%%%%%%%%%%%%%%%%% LyX specific LaTeX commands.
\providecommand{\LyX}{L\kern-.1667em\lower.25em\hbox{Y}\kern-.125emX\@}
%% A simple dot to overcome graphicx limitations
\newcommand{\lyxdot}{.}


%%%%%%%%%%%%%%%%%%%%%%%%%%%%%% Textclass specific LaTeX commands.
\newenvironment{lyxcode}
{\begin{list}{}{
\setlength{\rightmargin}{\leftmargin}
\setlength{\listparindent}{0pt}% needed for AMS classes
\raggedright
\setlength{\itemsep}{0pt}
\setlength{\parsep}{0pt}
\normalfont\ttfamily}%
 \item[]}
{\end{list}}

\usepackage{babel}
\makeatother
\begin{document}

\title{TACPPD Developer Guide}

\maketitle
\begin{lyxcode}
~~~~~~~~~~~~~~~~~~~~~~~~~\includegraphics[bb = 0 0 200 100, draft, type=eps]{tacppd.org.jpg}
\end{lyxcode}

\author{(c) Copyright in 2001-2006 by}


\author{Roman Volkov and Brett Maxfield}


\section{Contacts}

See tacppd home site http://tacppd.org. For comments, issues and feature
requests you can use phorum or mailing list on tacppd project page
at SourceForge (http://sourceforge.net/projects/tacppd). For other
information you can contact directly with Roman Volkov, rv@tacppd.org.
For some extra information and services, commercial support and commercial
developing contact with http://tacppd.com.


\section{COPYRIGHT info}

The tacppd (tacacs++ server daemon) software is Copyright (C) 1998-2005
by Roman Volkov and tacppd team. You may use, distribute and copy
the tacppd software under the terms of GNU General Public License
version 2. See COPYING file in tacppd distribution for text of GNU
GPL. You should have received a copy of the GNU General Public License
along with this program; if not, write to the Free Software Foundation,
Inc., 59 Temple Place, Suite 330, Boston, MA 02111-1307 USA. This
product includes software developed by Eric Young. This product includes
software developed by Alec Peterson. It uses RSA Data Security, Inc.
MD5 Message-Digest Algorithm. Also some copyright information for
other parts available inside source code (NET-SNMP library code).


\section{SUPPORT info}

This product has community support, available via http://tacppd.org
resources: mailing list, phorum, bugtracking and feature requests
system from SourceForge's tacppd page (http://sourceforge.net/progects/tacppd).
You should know, that support will be provided only when community
people will have free time and possibilities to do it, so please,
don't require a lot. For information about commercial support visit
http://tacppd.com, but you should know, that support will not be provided
under any circumstances for this program by tacppd.com, it's employees,
volunteers or directors, unless a commercial support agreement and/or
explicit prior written consent has been given.


\section{Introduction}

Tacppd APIs, internal structures, programming style, project goals
and methods.


\section{Writing documentation}

For documentation writing we use \LyX{} (http://www.lyx.org). The
\LyX{} files are the source for both TEX and HTML. If you wish to
contribute toward the documentation, please do not edit changes directly
in the .lyx file but send your changes to me for merging. It is much
easier if you create a separate .lyx file for your documentation,
and for one of the maintainers can then merge it into the documentation.

On a daily basis i will generate TEX and HTML files. The new HTML
documents will be available on the web site shortly after. For program
code examples please use the {}``\LyX{}-code'' style.


\section{Developing environment}

We are currently using KDevelop. The KDevelop project file available
in distribution. The main platform - Linux.


\section{Programming style}

The main goal is to improve readability of code and thereby the understanding,
maintainability and general quality of the code. It is impossible
to cover all the specific cases as a general guide and the programmer
should be flexible and consoder that other people have to decypher
the code they write.


\subsection{C++}


\subsubsection{Comments}


\subsubsection{Naming conventions}

\begin{itemize}
\item Names representing types must be in mixed case, starting with upper
case.
\end{itemize}
\begin{description}
\item [{Example:}]~
\end{description}
\begin{lyxcode}
Line,~SavingsAccount
\end{lyxcode}
\begin{description}
\item [{Reason:}] Common practice in the C++ development community.
\end{description}
\begin{itemize}
\item Variable names must be in mixed case starting with lower case.
\end{itemize}
\begin{description}
\item [{Example:}]~
\end{description}
\begin{lyxcode}
line,~savingsAccount
\end{lyxcode}
\begin{description}
\item [{Reason:}] Common practice in the C++ development community.
\end{description}
\begin{itemize}
\item Named constants (including enumeration values) must be all uppercase
using underscore to separate words.
\end{itemize}
\begin{description}
\item [{Example:}]~
\end{description}
\begin{lyxcode}
MAX\_ITERATIONS,~COLOR\_RED,~PI
\end{lyxcode}
\begin{description}
\item [{Reason:}] Common practice in the C++ development community
\end{description}
\begin{itemize}
\item Names representing methods or functions must be verbs and written
in mixed case starting with lower case.
\end{itemize}
\begin{description}
\item [{Example:}]~
\end{description}
\begin{lyxcode}
getName(),~computeTotalWidth()
\end{lyxcode}
\begin{description}
\item [{Reason:}] Common practice in the C++ development community.
\end{description}
\begin{itemize}
\item Names representing namespaces should be all lowercase.
\end{itemize}
\begin{description}
\item [{Example:}]~
\end{description}
\begin{lyxcode}
analyzer,~iomanager,~mainwindow
\end{lyxcode}
\begin{description}
\item [{Reason:}] Common practice in the C++ development community.
\end{description}
\begin{itemize}
\item Names representing template types should be a single uppercase letter.
\end{itemize}
\begin{description}
\item [{Example:}]~
\end{description}
\begin{lyxcode}
template<class~T>~...~template<class~C,~class~D>~...~
\end{lyxcode}
\begin{description}
\item [{Reason:}] Common practice in the C++ development community.
\end{description}
\begin{itemize}
\item All names should be written in english.
\end{itemize}
\begin{description}
\item [{Example:}]~
\end{description}
\begin{lyxcode}
fileName;~//~NOT:~filNavn~
\end{lyxcode}
\begin{description}
\item [{Reason:}] English is the prefered language for international development.
\end{description}
\begin{itemize}
\item Abbreviations and acronyms must not be uppercase when used as name
\end{itemize}
\begin{description}
\item [{Example:}]~
\end{description}
\begin{lyxcode}
exportHtmlSource();~//~NOT:~exportHTMLSource();
\end{lyxcode}
\begin{description}
\item [{Reason:}] When the name is connected to another, the readbility
is seriously reduced; the word following the abbreviation does not
stand out as it should.
\end{description}
\begin{itemize}
\item Global variables should always be referred to using the :: operator.
\end{itemize}
\begin{description}
\item [{Example:}]~
\end{description}
\begin{lyxcode}
::mainWindow.open(),~::applicationContext.getName()
\end{lyxcode}
\begin{description}
\item [{Reason:}] For doing global variables separated from other.
\end{description}
\begin{itemize}
\item Private class variables should have underscore suffix.
\end{itemize}
\begin{description}
\item [{Example:}]~
\end{description}
\begin{lyxcode}
class~SomeClass~\{

~private:~

~~int~length\_;

~public:

\};
\end{lyxcode}
\begin{description}
\item [{Reason:}] This is important because class variables are considered
to have higher significance than method variables, and should be treated
with special care by the programmer.
\end{description}
\begin{itemize}
\item Variables with a large scope should have long names, variables with
a small scope can have short names
\end{itemize}
\begin{description}
\item [{Example:}]~
\end{description}
\begin{lyxcode}
i,~j,~k,~::objectForAllProject;
\end{lyxcode}
\begin{description}
\item [{Reason:}] Scratch variables used for temporary storage or indices
are best kept short
\end{description}
\begin{itemize}
\item The name of the object is implicit, and should be avoided in a method
name. 
\end{itemize}
\begin{description}
\item [{Example:}]~
\end{description}
\begin{lyxcode}
line.getLength();~//~NOT:~line.getLineLength();
\end{lyxcode}
\begin{description}
\item [{Reason:}] The latter seems natural in the class declaration, but
proves superfluous in use, as shown in the example. 
\end{description}

\subsubsection{TABs and spaces}

Special characters like TAB and page break must be avoided. (These
characters are bound to cause problem for editors, printers, terminal
emulators or debuggers when used in a multi-programmer, multi-platform
environment.)


\subsubsection{Include files and include statements}

\begin{itemize}
\item Header files must include a construction that prevents multiple inclusion.
The convention is an all uppercase construction of the file name with
\_ seperator and the h suffix with \_\_ before and after.
\end{itemize}
\begin{description}
\item [{Example:}] for file CoreBundle.h
\end{description}
\begin{lyxcode}
\#ifndef~\_\_CORE\_BUNDLE\_H\_\_

\#define~\_\_CORE\_BUNDLE\_H\_\_

~...

<code>

~...

\#endif~//\_\_CORE\_BUNDLE\_H\_\_
\end{lyxcode}
\begin{itemize}
\item Include statements must be located at the top of a file only.
\end{itemize}

\subsubsection{use spaces in type definitions}

Use only one space in type definition:

\begin{lyxcode}
int~intValue;

char~charValue;

int~intVal1,intVal2;
\end{lyxcode}
Use one space before {}``{*}'' in pointer definition:

\begin{lyxcode}
char~{*}strType;

int~{*}aaa;

void

methodName(int~a,char~{*}aaa,int~{*}bbb)~\{\}
\end{lyxcode}

\subsubsection{if() rule}

The space after last {}``)'' before {}``\{'', spaces around condition,
no space after {}``if'':

\begin{lyxcode}
if(aaa~>~bbb)~\{

~~~...

~~<code>

~~~...

\}
\end{lyxcode}
If use united conditions (with {}``\&\&'' and/or {}``\textbar{}\textbar{}''),
than no spaces around condition:

\begin{lyxcode}
if(aaa>bbb~\&\&~ccc<ddd)~\{

~~~...

~~<code>

~~~...

\}
\end{lyxcode}

\subsubsection{switch/case}

No space before first {}``('', one space after {}``)'' before
{}``\{'', {}``case'' moved for one space, code moved usually -
2 spaces. Comments about {}``case'' in one line with it.

\begin{lyxcode}
switch(statement)~\{

~case~1:~~//~case~comment

~~~~...

~~~<code>

~~~~...

~~~break;

~case~2:~~//~case~comment

~~~~...

~~~<code>

~~~~...

~~~break;

~default:~~//~default~comment

~~~~...

~~~<code>

~~~~...

~~~break;

\}
\end{lyxcode}

\subsubsection{method}

Method return value in separate line, no space after method name before
{}``('', no space after {}``,'' in method arguments, space after
{}``)'' and before {}``\{'', {}``\{'' in one line with method
name and arguments:

\begin{lyxcode}
void

CoreClass::methodX(int~input,char~{*}str)~\{

~~~...

~~<code>

~~~...

\}
\end{lyxcode}

\section{TACPPD operation}

The intent of tacppd operation is - use some OO model for provide
universal access to TCP/UDP server, database engine, configuration
engine, logging engine.


\section{Internals}


\section{Data structures}

As core of the project we use Core{*} obgects. It has all tacppd configuration
inside.


\section{Logging/Debug system}

In main.cc during tacppd start you can see creation of next object:
logTermSystem, logSyslogSystem, logSystem.


\subsection{log/debug facilities}

There are several log/debug facilities available (the latest list
you can see in sources Debug.cc/Debug.h):

\begin{lyxcode}
//~Debug~Facilities~1-250

\#define~LOG\_TACACS~~~~~1~~~~~~~~~~~~~~~//~tacacs+~protocol~events

\#define~LOG\_BILLING~~~~2~~~~~~~~~~~~~~~//~billing~subsytem

\#define~LOG\_AAA~~~~~~~~3~~~~~~~~~~~~~~~//~AAA

\#define~LOG\_DB~~~~~~~~~4~~~~~~~~~~~~~~~//~database~system~activity

\#define~LOG\_SNMP~~~~~~~5~~~~~~~~~~~~~~~//~snmp~system~activity

\#define~LOG\_NETFLOW~~~~6~~~~~~~~~~~~~~~//~netflow~collector

\#define~LOG\_SERVER~~~~~7~~~~~~~~~~~~~~~//~tcp/udp~server

\#define~LOG\_CC~~~~~~~~~8~~~~~~~~~~~~~~~//~cryptocard

\#define~LOG\_POOL~~~~~~~9~~~~~~~~~~~~~~~//~address~pool~system

\#define~LOG\_TRACK~~~~~~10~~~~~~~~~~~~~~//~user~track~system

\#define~LOG\_HTTP~~~~~~~11~~~~~~~~~~~~~~//~http~interface

\#define~LOG\_TIP~~~~~~~~12~~~~~~~~~~~~~~//~bundle~protocol

\#define~LOG\_TEP~~~~~~~~13~~~~~~~~~~~~~~//~external~protocol
\end{lyxcode}
Also there are several internal facilities for output to logfiles
(defined in Logging.h file):

\begin{lyxcode}
//~log~critical~error~messages

\#define~LOG\_ERROR~~~~~~~~~~~~~~~~~0

//~log~NetFlow

\#define~LOG\_NETFLOWLOG~~~~~~~~~~~~252

//~log~events

\#define~LOG\_EVENT~~~~~~~~~~~~~~~~~253

//~log~informational

\#define~LOG\_INFORM~~~~~~~~~~~~~~~~254

//~log~logging~information~to~other~file

\#define~LOG\_LOGGING~~~~~~~~~~~~~~~255
\end{lyxcode}

\subsection{logTermSystem}

During new terminal session creation, you should register socket descriptor
of this session with this system by use call:

\begin{lyxcode}
::logTermSystem->add(<socket~handler>);
\end{lyxcode}
It means, that logging system will have information for output to
terminal. Also when you close terminal session, do not forget call:

\begin{lyxcode}
::logTermSystem->del(<socket~handler>);
\end{lyxcode}
This system periodic checks list of messages and if any created entry
require present facility, than it send this string to socket. To require
any facility you can use call:

\begin{lyxcode}
::logTermSystem->add\_debug\_fc(<socket~handler>,<facility~number>);
\end{lyxcode}
For close output of special facility use:

\begin{lyxcode}
::logTermSystem->del\_debug\_fc(<socket~handler>,<facility~number>);
\end{lyxcode}

\section{Database API}

The database system API was created for provide possible universal
access to any database engine of data file. It has loadable modules-drivers,
which should provide low-level access. All database structures uses
tacppd point of view. It means, that we have some fixed set of types
which can be not supported by database engine, but database support
module must accept this types and convert it and data to database
appropriate form. 


\subsection{Database version 1}

The version 1 of tacppd database currently only version.


\subsubsection{users data}

\begin{description}
\item [{Name:}] tacpp\_v1\_usr
\end{description}
\begin{lyxcode}
user~name~(username)~-~string

password~(password)~-~string

open~date~(dopen)~-~datatime

close~date~(dclose)~-~datatime

authorization~group~(avpid)~-~string

access~group~(accid)~-~string

resource~group~(bilid)~-~string

additional~data~group~(addid)~-~string

max~sessions~(maxsess)~-~integer
\end{lyxcode}
This is information for authentication. Store username, password,
open/close date and time, several groups membership (access group,
authorization group, additional user data group), maximum allowed
sessions. If database key not equal word {}``none'', than field
password crypted. It crypted with use md5 and than stored in base64.

EXAMPLE:

- lets create user {}``ppprv'' with password {}``test'', authorization
group {}``ppp-1'', access group {}``ppp-acc'', no resource group
and maximum 1 session available.

\begin{lyxcode}
username=ppprv

password=test

avpid=ppp-1

accid=ppp-acc

bilid=

maxsess=1
\end{lyxcode}

\subsubsection{additions to users data}

\begin{description}
\item [{Name:}] tacpp\_v1\_add
\end{description}
\begin{lyxcode}
additional~data~group

authentication~source

enable~access~password
\end{lyxcode}

\subsubsection{user access rights}

\begin{description}
\item [{Name:}] tacpp\_v1\_acc
\end{description}
\begin{lyxcode}
access~group~name~(accid)

access~by~phone

from~specifyed~NAS

from~specifyed~port

restricted~access~time


\end{lyxcode}

\subsubsection{authorization information}

\begin{description}
\item [{Name:}] tacpp\_v1\_avp
\end{description}
\begin{lyxcode}
author~group~id

service

protocol

attribute=vlue~pair
\end{lyxcode}

\subsubsection{per-command access authorization}

\begin{description}
\item [{Name:}] tacpp\_v1\_cmd
\end{description}
\begin{lyxcode}
author~group~id~(avpid)

permitted~commands

denied~commands

permitted~arguments

denied~arguments
\end{lyxcode}

\subsubsection{resource module data}

\begin{description}
\item [{Name:}] tacpp\_v1\_res
\end{description}

\subsubsection{resource module resources information}

\begin{description}
\item [{Name:}] tacpp\_v1\_resdata
\end{description}

\subsubsection{resource changes log}

\begin{description}
\item [{Name:}] tacpp\_v1\_reslog
\end{description}

\subsection{Database structure definition in tacppd database API}

In files TacDb.cc and TacDb.h++

TacDb.h++: in class TacDb in provate section you can find or add database
table descriptor as: DbTable {*}table\_name\_var;

TacDb.cc: in constructor you should add: table\_name\_var = new DbTable(\char`\"{}table\_name\char`\"{},\char`\"{}table
description\char`\"{}); and also add table fields: 

table\_name\_var-\textgreater{}add(\char`\"{}field\_name\char`\"{},\char`\"{}field
description\char`\"{},FIELD\_TYPE); where FIELD\_TYPE = STR\_TYPE
- char(), varchar() - string 

for this type you should add field size:

table\_name\_var-\textgreater{}add(\char`\"{}field\_str\char`\"{},\char`\"{}description\char`\"{},STR\_TYPE,32); 

UINT\_TYPE - unsigned int 

INT\_TYPE - ordinary int (remember, usually in SQL databases no signed
int type, and this will unsigned too)

FLOAT\_TYPE - float

DATE\_TYPE - datetime or similar 

BOOL\_TYPE - boolean


\section{Database module}

Database modules system was created for provide acceess to any database
engine. Any module have to be linked with database library. We have
untested feature for reload one type database module to other during
operation, also for change module version or for change database library
version without tacppd restart.


\section{SNMP module}

Snmp modules system was created for provide easy implementation snmp
operation with any snmp-capable device. We use snmp operation with
net-snmp library (www.net-snmp.org). The snmp low-level code situated
inside tacppd core. The modules are only interpret results.


\section{Resource module}

This is one method for implement resource control system.


\section{XML-RPC}


\subsection{XML-RPC server side implementation}


\subsection{XML-RPC client side implementation}


\subsubsection{C/C++}


\subsubsection{Perl}


\subsubsection{Java}


\section{Debugging}


\subsection{memory leaks}

For memory operation debug we will use Valgrind (http://developer.kde.org/\textasciitilde{}sewardj/)
- an open-source memory debugger for x86-GNU/Linux.


\section{Testing}

Some TACACS+ testing tools available with libtacplus library http://sf.net/projects/libtacplus
in samples directory (tacacs+ client, etc)
\end{document}
